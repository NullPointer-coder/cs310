\documentclass[11pt]{article}
\newcommand{\name}{Jingbo Wang} % <=== PUT YOUR NAME HERE

\usepackage[paper=letterpaper, margin=1in, headheight=13.6pt]{geometry}
\usepackage{fancyhdr}
\pagestyle{fancy}
\fancyhf{}
\rhead{\name{}}
\cfoot{Page \thepage}

\usepackage[parfill]{parskip}
\usepackage{amsmath}
\usepackage{graphicx}
\usepackage{fancyvrb}
\usepackage{upquote}

\newcommand{\problem}[1]{\vspace*{2ex}\textbf{Problem #1 ---} }
\newcommand{\answer}{\textit{Answer: }}

\begin{document}
\thispagestyle{empty}

\begin{center}
{\large CS 310}\\
Assignment 113\\
\today
\end{center}

\begin{flushright}
\name{}
\end{flushright}

\problem{1} Algorithms often have the following properties:

\begin{itemize}
\item the steps are stated \emph{unambiguously} so that there is
  no question how the algorithm proceeds
\item the algorithm is \emph{deterministic} so that repeating the
  algorithm on the same input produces the same output
\item the algorithm is \emph{finite} because it terminates after a
  finite number of steps have been performed
\item the algorithm produces \emph{correct} output for a given input
\end{itemize}

For the following algorithm, for each property listed above, determine
whether the algorithm exhibits this property:

\begin{Verbatim}[numbers=left,xleftmargin=5mm]
unsigned max3(unsigned a, unsigned b, unsigned c)
{
  unsigned result = a;
  if (b > result)
  {
    result = b;
  }
  if (c > result)
  {
    result = c;
  }
  return result;
}
\end{Verbatim}

\answer This algorithm is unambiguous because the syntax for the
operations is well-understood.  It is deterministic because it always
produces the same output for a given input.  It is finite because the
number of lines of code executed (including the header) is strictly
between 3 and 7 inclusive.  It is correct because for all possible
valid input combinations it does in fact return a value equal to the
maximum input value.

\problem{2} Repeat problem 1 for the following algorithm.  This
algorithm empirically checks the correctness of Goldbach's conjecture,
which states (in a modern interpretation) that every even number
greater than 2 is the sum of two prime numbers.  Assume
\verb.has_prime_addends. is a valid function that correctly determines
whether its argument has two prime addends.

\begin{Verbatim}[numbers=left,xleftmargin=5mm]
bool goldbach()
{
  unsigned value = 4;
  bool ok = true;
  while (ok)
  {
    if (!has_prime_addends(value))
    {
      ok = false;
    }
    else
    {
      value += 2;
    }
  }
  return ok;
}
\end{Verbatim}

\answer This algorithm is unambiguous because the syntax for the
operations is well-understood. It is not deterministic because it does not
produce any output for a given input. It is not finite because the algorithm only runs lines 11 - 14 in the while loop, and  "ok" value is always true, it cannot stop. It is not correct because this algorithm cannot give any Boolean value.

\problem{3} What is the hexadecimal representation of $724_{10}$?

\answer The first few powers of 16 are:

\begin{align*}
16^0 &= 1\\
16^1 &= 16\\
16^2 &= 256\\
16^3 &= 4096\\
\end{align*}

Thus we have:

\begin{equation*}
\begin{split}
724&\\
\underline{-2 \times 256 = 512}&\\
212&\\
\underline{-13 \times 16 = 208}&\\
4&\\
\underline{-4 \times 1 = 4}&\\
0&
\end{split}
\end{equation*}

And thus we have $724_{10} = 2d4_{16}$.

\problem{4} Based on the hexadecimal value found in the previous
solution, what is the binary representation of $724_{10}$?

\answer Because we have $2d4_{16}$, it is easy to know that:

\begin{align*}
2_{16} &= 0010_{2}\\
d_{16} &= 1101_{2}\\
4_{16} &= 0100_{2}\\
\end{align*}

Thus we have:

\begin{equation*}
\begin{split}
2d4_{16} = 001011010100_{2}
\end{split}
\end{equation*}

And thus we have $724_{10} = 001011010100_{2}$.

\problem{5} What is the decimal representation of \texttt{0x2b3a}?

\answer It is easy to know that:
\begin{equation*}
\begin{split}
a_{16} &= 10_{10}\\
b_{16} &= 11_{10}
\end{split}
\end{equation*}

Thus we have:

\begin{equation*}
\begin{split}
  &10 \times 16^0 + 3 \times 16^1 + 11 \times 16^2 + 2 \times 16^3\\
= &10 + 48 + 2816 + 8192\\
= &11066\\
\end{split}
\end{equation*}


\end{document}